% playlist: 1-7,1-7,1-7,1-7,1-8
\documentclass[a4paper,12pt]{bescript}

%opening
\title{Virada de Terceira}

\begin{document}
\maketitle

\begin{staffline}
4/4      &\mhead{\it virada de $3^a$} &\mhead{\it desenho} &\mhead{\it subida} &\mhead{\it levada} \\ \hline
%Agogo   &....&....&....&....&....&....&....&....&....&....&....&....&....&....&....&....\\
Chocalho &xxxx&x.x.&x...&....&xxx &x.. &xxx &x.. &xxx &xxx &xxx &xxx &x...&....&....&....\\
Tamborim &xxxx&x.x.&x...&....&xxx &x.. &xxx &x.. &xxx &xxx &xxx &xxx &x...&....&....&....\\
Caixa    &xxxx&x.x.&x...&....&xxx &x.. &xxx &x.. &/   &/   &/   &/   &/   &/   &/   &/   \\
Repi     &xxxx&x.x.&x...&....&xxx &x.. &xxx &x.. &/   &/   &/   &/   &/   &/   &/   &/   \\
Primeira &....&..x.&....&..x.&....&..x.&....&..x.&....&x...&....&x...&....&x...&....&x...\\
Segunda  &x...&....&x...&....&x...&....&x...&....&x...&....&x...&....&x...&....&x...&....\\
Terceira &....&..x.&....&..x.&....&..x.&....&..x.&..x.&..x.&..x.&..x.&..x.&..x.&..x.&xx.x\\
\hline
\end{staffline}

%\begin{staffline}
%         &\mhead{} &\mhead{}           &\mhead{}           &\mhead{\it virada e fim} \\ \hline
%Agogo   &....&....&....&....&....&....&....&....&....&....&....&....&....&....&....&....\\
%Chocalho &/   &/   &/   &/   &/   &/   &/   &/   &/   &/   &/   &/   &xxxx&x.x.&x...&....\\
%Tamborim &/   &/   &/   &/   &/   &/   &/   &/   &/   &/   &/   &/   &xxxx&x.x.&x...&....\\
%Caixa    &/   &/   &/   &/   &/   &/   &/   &/   &/   &/   &/   &/   &xxxx&x.x.&x...&....\\
%Repi     &/   &/   &/   &/   &/   &/   &/   &/   &/   &/   &/   &/   &xxxx&x.x.&x...&....\\
%Primeira &....&x...&....&x...&....&x...&....&x...&....&x...&....&x...&....&..x.&....&....\\
%Segunda  &x...&....&x...&....&x...&....&x...&....&x...&....&x...&....&x...&....&x...&....\\
%Terceira &....&x.x.&....&xx.x&....&x.x.&..x.&xx.x&....&x.x.&....&xx.x&....&..x.&....&....\\
%Apito   &....&....&....&....&....&....&....&....&x...&x...&x...&x...&....&....&....&....\\
%\hline
%\end{staffline}

\section*{Anmerkungen}
Die {\it virada de terceira} (dt. dritter Abriss) ist eine Alternative, um in den ersten Teil des Enredos zu führen. Zwischen der \kw{virada} und der \kw{subida} wird ein weiterer Takt eingefügt, indem eine Konvention gespielt wird.

\noindent Folgende Punkte sind zu beachten:
\begin{enumerate}


 \item Die {\it virada} wird immer angezeigt. Die häufigsten Form sie anzuzeigen ist in XXX dargestellt: Eine Hand dreht () und die zweite zeigt eine Drei an (). Eine weitere, verbreitete Form ist mit einer oder mit beiden Händen eine Drei anzuzeigen und dann die Drehung durch eine spezielle Handbewegung einzuleiten.

 \item Auf die {\it virada} folgt ein \kw{desenhho} welches alle Instrumente spielen und eine \kw{subida}, wie sie nach der virada de primeira gespielt wird.
 \item Dieses \kw{desenho} variiert zwischen den Samba-Schulen. Am häufigsten werden Triolen (siehe Takt 2) oder Sechzehntel gespielt. Im folgenden Abschnitt werden gängige Variationen vorgestellt.

 \item Es gibt keine einheitliche Form für die Chocalhos. In einige Schulen sind sie stumm, in anderne folgen sie den Tamborins oder den Caixas.

\end{enumerate}

\section*{Variationen für das Desenho}
Es gibt keine einheitliche Form den zweiten Takt. Die drei häufigsten Formen sind die Triolen (siehen Takt 2), die Sechzehntel (Takt 9) oder mit einem Press-Roll (Takt 10).


\begin{staffline}
         &\mhead{Sechzehntel} &\mhead{Press-Roll}           &\mhead{}           &\mhead{} \\ \hline
%Agogo   &....&....&....&....&....&....&....&....&....&....&....&....&....&....&....&....\\
Chocalho &xxxx&x...&xxxx&x...&....&....&....&....&....&....&....&....&....&....&....&....\\
Tamborim &xxxx&x...&xxxx&x...&....&....&....&....&....&....&....&....&....&....&....&....\\
Caixa    &xxxx&x...&xxxx&x...&xZzl&x...&xZzl&x...&....&....&....&....&....&....&....&....\\
Repi     &xhho&x...&xxxx&x...&xZzo&h...&xZzo&h...&....&....&....&....&....&....&....&....\\
%Primeira &....&x...&....&x...&....&x...&....&x...&....&x...&....&x...&....&..x.&....&....\\
%Segunda  &x...&....&x...&....&x...&....&x...&....&x...&....&x...&....&x...&....&x...&....\\
%Terceira &....&x.x.&....&xx.x&....&x.x.&..x.&xx.x&....&x.x.&....&xx.x&....&..x.&....&....\\
%Apito   &....&....&....&....&....&....&....&....&x...&x...&x...&x...&....&....&....&....\\
\hline
\end{staffline}


%\section*{Variationen von der União da Ilha}
%https://youtu.be/YP0P8E0R7so?t=161




\end{document}
