% playlist: 1-7,1-7,1-7,1-7,1-8

\documentclass[a4paper,12pt]{bescript}
\renewcommand{\familydefault}{\ttdefault}

%opening
\title{Virada de Primeira}

\begin{document}
\maketitle


\begin{staffline}
4/4      &\mhead{\it virada $1^a$} &\mhead{\it subida}      &\mhead{}           &\mhead{\it levada} \\ \hline
%Agogo   &....&....&....&....&....&....&....&....&....&....&....&....&....&....&....&....\\
Chocalho &xxxx&x.x.&x...&....&....&....&....&....&....&....&....&....&/   &/   &/   &/   \\
Tamborim &xxxx&x.x.&x...&....&xxx &xxx &xxx &xxx &x   &....&....&....&/   &/   &/   &/   \\
Caixa    &xxxx&x.x.&x...&....&/   &/   &/   &/   &/   &/   &/   &/   &/   &/   &/   &/   \\
Repi     &xxxx&x.x.&x...&....&/   &/   &/   &/   &/   &/   &/   &/   &/   &/   &/   &/   \\
Primeira &....&..x.&....&..x.&....&x...&....&x...&....&x...&....&x...&....&x...&....&x...\\
Segunda  &x...&....&x...&....&x...&....&x...&....&x...&....&x...&....&x...&....&x...&....\\
Terceira &....&..x.&....&..x.&..x.&..x.&..x.&..x.&..x.&..x.&..x.&xx.x&....&x.x.&....&xx.x\\
\hline
\end{staffline}

%\begin{staffline}
%         &\mhead{} &\mhead{}           &\mhead{}           &\mhead{\it virada e fim} \\ \hline
%Agogo   &....&....&....&....&....&....&....&....&....&....&....&....&....&....&....&....\\
%Chocalho &/   &/   &/   &/   &/   &/   &/   &/   &/   &/   &/   &/   &xxxx&x.x.&x...&....\\
%Tamborim &/   &/   &/   &/   &/   &/   &/   &/   &/   &/   &/   &/   &xxxx&x.x.&x...&....\\
%Caixa    &/   &/   &/   &/   &/   &/   &/   &/   &/   &/   &/   &/   &xxxx&x.x.&x...&....\\
%Repi     &/   &/   &/   &/   &/   &/   &/   &/   &/   &/   &/   &/   &xxxx&x.x.&x...&....\\
%Primeira &....&x...&....&x...&....&x...&....&x...&....&x...&....&x...&....&..x.&....&....\\
%Segunda  &x...&....&x...&....&x...&....&x...&....&x...&....&x...&....&x...&....&x...&....\\
%Terceira &....&x.x.&....&xx.x&....&x.x.&..x.&xx.x&....&x.x.&....&xx.x&....&..x.&....&....\\
%Apito   &....&....&....&....&....&....&....&....&x...&x...&x...&x...&....&....&....&....\\
%\hline
%\end{staffline}


\section*{Anmerkungen}
Die {\it virada de primeira} (dt. Einser-Abriss) ist die häufigste {\it paradinha}. Sie leitet immer in den ersten Teil des {\it samba enredo}. Wenn nach der {\it virada} nicht gestoppt wird, folgt eine {\it subida} (dt. Aufgang) der einzelnen instrumente.
Die hier gezeigte {\it virada} stellt sie version dar, die wir in Bloco Esperança spielen. Viele Samba Schulen haben eingen Anpasusngen entwickelt.

Folgende Punkte sind zu beachten:
\begin{enumerate}
 \item Die {\it virada} wir immer angezeigt. Die häufigsten Form sie anzuzeigen ist in XXX dargestellt: Eine Hand dreht () und die zweite zeigt eine Eins an (). Eine weitere, verbreitete Form ist mit einer oder mit beiden Händen eine Eins anzuzeigen und dann die Drehung durch eine spezielle Handbewegung einzuleiten.

 \item Auf die {\it virada} folgt ein Aufgang ({\it subida}) für Chocalhos, Tamborins und Surdo de Primeira (Vorschlag) und Terceira. Im Anschluss gehen alle Instrumente in die {\it levada} über.

 \item Grundsätzlich wird der Abriss  immer durch alle Instruemnte gepielt, wenn diese zuvor pausierten. Ausnahmen hierzu werden angezeigt oder ergeben sich durch das {\it desenho}.


\end{enumerate}


\end{document}
