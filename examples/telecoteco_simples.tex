\documentclass[a4paper,12pt]{article}
\renewcommand{\familydefault}{\ttdefault}
\usepackage{graphicx}
\usepackage[utf8]{inputenc}
\usepackage{url}
\usepackage{colortbl,xcolor}
\usepackage[top=25mm,bottom=15mm,right=15mm,left=15mm]{geometry}
\usepackage{fancyhdr}

\usepackage[colorlinks,linkcolor={black},citecolor={black},urlcolor={black}]{hyperref}

\fancypagestyle{plain}{%
  \renewcommand{\headrulewidth}{0pt}%
  \fancyhf{}%
  \fancyfoot[C]{\begin{picture}(0,0) \put(-10,0){%
  %\includegraphics[width=1cm]{logo} %
  } \end{picture}\footnotesize}%
  \fancyfoot[R]{\thepage}%
}


\lfoot{\begin{picture}(0,0) \put(-10,0){\includegraphics[width=1cm]{logo}} \end{picture}}
\rfoot{\thepage}

%opening
\title{Telecoteco Simples}
\author{}
\date{}

\definecolor{Gray}{gray}{0.85}

\newcounter{measure}
\setcounter{measure}{1}
\def\m{\themeasure\stepcounter{measure}}
\newcommand{\mhead}[1]{\multicolumn{4}{l|}{\m{} #1}}

\newcolumntype{A}[0]{>{}p{5em}}
\newcolumntype{C}[0]{>{\centering\arraybackslash}p{2.5em}}
\newcolumntype{D}[0]{>{\columncolor{Gray}\centering\arraybackslash}p{2em}}


\newenvironment{staffline}{\begingroup
\small
\setlength{\tabcolsep}{0pt} % Default value: 6pt
%^\renewcommand{\arraystretch}{1.5} % Default value: 1
\begin{tabular}{A|CDCD|CDCD|CDCD|CDCD|}
}{
\end{tabular}
\endgroup
\vspace{1ex}
}
\sloppy

\begin{document}
\pagestyle{plain}
\maketitle

\begin{staffline}
4/4      &\mhead{}           &\mhead{}           &\mhead{}           &\mhead{} \\ \hline
Chocalho &....&....&....&....&....&....&....&....&/   &/   &/   &/   &/   &/   &/   &/   \\
Tamborim &....&....&....&....&....&....&....&....&x.x.&x.x.&x.x.&xx.x&.x.x&..x.&x.x.&x..x\\
Repi     &x.h.&x.h.&cccc&cccc&/   &/   &/   &/   &/   &/   &/   &/   &/   &/   &/   &/   \\
Caixa    &....&....&....&....&....&/   &/   &/   &/   &/   &/   &/   &/   &/   &/   &/   \\
Primeira &....&....&....&....&....&x...&....&x...&....&x...&....&x...&....&x...&....&x...\\
Segunda  &....&....&....&....&....&....&x...&....&x...&....&x...&....&x...&....&x...&....\\
Terceira &....&....&....&....&....&x.x.&..x.&..x.&..x.&..x.&..x.&xx.x&....&x.x.&....&xx.x\\
\end{staffline}

\begin{staffline}
         &\mhead{}           &\mhead{}           &\mhead{}           &\mhead{} \\ \hline
Chocalho &/   &/   &/   &/   &/   &/   &/   &/   &/   &/   &/   &/   &xxxx&x.x.&x...&....\\
Tamborim &.x.x&..x.&x.x.&x..x&.x.x&..x.&x.x.&x..x&.x.x&..x.&x.x.&x...&xxxx&x.x.&x...&....\\
Repi     &/   &/   &/   &/   &/   &/   &/   &/   &/   &/   &/   &/   &xxxx&x.x.&x...&....\\
Caixa    &/   &/   &/   &/   &/   &/   &/   &/   &/   &/   &/   &/   &xxxx&x.x.&x...&....\\
Primeira &....&x...&....&x...&....&x...&....&x...&....&x...&....&x...&....&..x.&....&....\\
Segunda  &x...&....&x...&....&x...&....&x...&....&x...&....&x...&....&x...&....&x...&....\\
Terceira &....&x.x.&..x.&xx.x&....&x.x.&..x.&..x.&..x.&..x.&..x.&xx.x&....&..x.&....&....\\
Apito    &x...&....&....&....&x...&....&....&....&x...&x...&x...&x...&....&....&....&....\\ \hline
\end{staffline}

\section*{Anmerkungen}
\begin{itemize}
 \item Telecoteco ab Takt 4, wobei der Auftakt 3 immer gespielt wird. 
 In dieser vereinfachten Version werden die beiden Doppelschläge weggelassen.  
 \item Der Auftakt (Takt 3) besteht aus 6 Achtzehntel.
 \item Beider Drehung (virada), stoppen die Tamborims auf der 4. Viertel (Siehe Takt 7, 4. Zählzeit).
\end{itemize}

\section*{Referenzen}
\footnotesize
\begin{itemize}
 \item \url{https://youtu.be/EictRYyBzcA}
 \item \url{https://youtu.be/XPsC2085AQQ}
\end{itemize}



\end{document}
