% playlist: 1-7,1-7,1-7,1-7,1-8

\documentclass[a4paper,12pt]{bescript}


%opening
\title{Virada de Segunda}

\begin{document}
%\pagestyle{plain}
\maketitle

\begin{staffline}
4/4      &\mhead{\it virada de $2^a$} &\mhead{\it subida}      &\mhead{}           &\mhead{\it levada} \\ \hline
%Agogo   &....&....&....&....&....&....&....&....&....&....&....&....&....&....&....&....\\
Chocalho &xxxx&x.x.&x...&....&....&....&....&....&....&....&....&....&....&....&....&....\\
Tamborim &xxxx&x.x.&x...&....&x.x.&x.x.&x.x.&xx.x&.x.x&x.x.&x.x.&xx.x&.x.x&x.x.&x.x.&xx.x\\
Caixa    &xxxx&x.x.&x...&....&/   &/   &/   &/   &/   &/   &/   &/   &/   &/   &/   &/   \\
Repi     &xxxx&x.x.&x...&....&/   &/   &/   &/   &/   &/   &/   &/   &/   &/   &/   &/   \\
Primeira &....&..x.&....&..x.&....&x...&....&x...&....&x...&....&x...&....&x...&....&x...\\
Segunda  &x...&....&x...&....&x...&....&x...&....&x...&....&x...&....&x...&....&x...&....\\
Terceira &....&..x.&....&..x.&..x.&..x.&..x.&..x.&..x.&..x.&..x.&xx.x&....&x.x.&....&xx.x\\
\hline
\end{staffline}

\begin{staffline}
         &\mhead{} &\mhead{}           &\mhead{}           &\mhead{\it virada e fim} \\ \hline
%Agogo   &....&....&....&....&....&....&....&....&....&....&....&....&....&....&....&....\\
Chocalho &....&....&....&....&....&....&....&....&....&....&....&....&xxxx&x.x.&x...&....\\
Tamborim &.x.x&x.x.&x.x.&xx.x&.x.x&x.x.&x.x.&xx.x&.x.x&x.x.&x.x.&x...&xxxx&x.x.&x...&....\\
Caixa    &/   &/   &/   &/   &/   &/   &/   &/   &/   &/   &/   &/   &xxxx&x.x.&x...&....\\
Repi     &/   &/   &/   &/   &/   &/   &/   &/   &/   &/   &/   &/   &xxxx&x.x.&x...&....\\
Primeira &....&x...&....&x...&....&x...&....&x...&....&x...&....&x...&....&..x.&....&....\\
Segunda  &x...&....&x...&....&x...&....&x...&....&x...&....&x...&....&x...&....&x...&....\\
Terceira &....&x.x.&....&xx.x&....&x.x.&..x.&xx.x&....&x.x.&....&xx.x&....&..x.&....&....\\
%Apito   &....&....&....&....&....&....&....&....&x...&x...&x...&x...&....&....&....&....\\
\hline
\end{staffline}

\section*{Anmerkungen}
Die {\it virada de segunda} (dt. Zweier-Abriss) ist eine häufige {\it paradinha}. Sie leitet immer in den zweiten Teil des {\it samba enredo}. Im zweiten Teil spielen die Tamborins, Agogos und Chocalhos in der Regel ein {\it desenho}. Dieser {\it desenho} wird für jeden Enredo neu komponiert. In Bloco Esperança spielen die Tamborins den {\it telecoteco} und die Chocalhos und Agogos haben eine Pause. In vielen anderen Hobby-Baterias pausieren auch die Tamborins.

\noindent Folgende Punkte sind zu beachten:
\begin{enumerate}


 \item Die {\it virada} wir immer angezeigt. Die häufigsten Form sie anzuzeigen ist in XXX dargestellt: Eine Hand dreht () und die zweite zeigt eine Zwei an (). Eine weitere, verbreitete Form ist mit einer oder mit beiden Händen eine Zwei anzuzeigen und dann die Drehung durch eine spezielle Handbewegung einzuleiten.

 \item Auf die {\it virada} (Takt 1) folgt ein Aufgang ({\it subida}, Takt 2) für Tamborins, die dann in den telecoteco gehen (ab Takt 3).

 \item Die Chocalhos und Agogos spielen nicht. Wichtig ist aber zu beachten, dass alle Instrumente mit dem nächsten Abriss einsetzen. In andern Worten: Der Abriss wird immer durch alle Instrumente gespielt.
 \item Die Küche spielt wie bei einem Einser-Abriss. Doch weil Chocalhos fehlen, sollten die Caixas lauter spielen.
\end{enumerate}


\end{document}
